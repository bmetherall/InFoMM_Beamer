\documentclass{beamer}

\usetheme{InFoMM}

\title{A New Method of Modelling Tuneable Lasers with Functional Composition}
\date{October 2019}
\author{Brady Metherall}

\begin{document}
\frame{\titlepage}

\section{Introduction}

\frame{
\frametitle{Laser Overview}
\begin{itemize}
\item The word laser was originally an acronym for Light Amplification by Stimulated Emission of Radiation
\item The defining feature of laser light is coherence, where the peaks and trough overlap causing very strong constructive interference
\item Typical lasers, such as Helium-Neon gas lasers or laser pointers, are monochromatic (operate at a single wavelength---have a very narrow bandwidth)
\end{itemize}
}

\frame{
\frametitle{Tuneable Lasers}
Tuneable lasers
\begin{itemize}
\item have a much wider bandwidth (up to $\sim 100$ nm)
\item lase continuously at all of these wavelengths
\item have applications in spectroscopy and high resolution imaging
%\item use mode-locking---a method to create ultrashort laser pulses ($10^{-12}$--$10^{-15}$ s)
\end{itemize}
}

\section{Modelling Efforts}

\frame{
\frametitle{Nonlinear Optics}
The standard equation for studying nonlinear optics is the generalized nonlinear Schr\"odinger equation,
\begin{align}
\pdiff{A}{z} &= \underbrace{- i \frac{\beta_2}{2}\pdiff[2]{A}{T}}_{\text{Dispersion}} + \underbrace{i \gamma |A|^2 A}_{\text{Nonlinearity}} + \underbrace{\underbrace{\frac{1}{2}g(A) A}_{\text{Gain}} - \underbrace{\alpha A}_{\text{Loss}}}_{\text{Generalization}}.
\label{eq:gnlse}
\end{align}
\begin{itemize}
\item Derived from the nonlinear wave equation
\item Uses comoving coordinates so that the reference frame propagates with the pulse at the group velocity
\begin{align*}
T = t - \frac{z}{v_g}
\end{align*}
\end{itemize}
}

\end{document}