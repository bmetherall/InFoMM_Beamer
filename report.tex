\documentclass[8pt]{beamer}

\usepackage{graphicx}
\usepackage{ragged2e}   


\usefonttheme{serif}

\justifying
\addtobeamertemplate{block begin}{}{\justifying}

\hbadness=1000 % Supresses bad box warnings

% Gives the nicer SQRT symbol.
\usepackage{letltxmacro} 
\makeatletter
\let\oldr@@t\r@@t
\def\r@@t#1#2{%
	\setbox0=\hbox{$\oldr@@t#1{#2\,}$}\dimen0=\ht0
	\advance\dimen0-0.2\ht0
	\setbox2=\hbox{\vrule height\ht0 depth -\dimen0}%
	{\box0\lower0.4pt\box2}}
\LetLtxMacro{\oldsqrt}{\sqrt}
\renewcommand*{\sqrt}[2][\ ]{\oldsqrt[#1]{#2}}
\makeatother


% Makes beamer lists nicely justified
\makeatletter
\renewcommand{\itemize}[1][]{%
	\beamer@ifempty{#1}{}{\def\beamer@defaultospec{#1}}%
	\ifnum \@itemdepth >2\relax\@toodeep\else
	\advance\@itemdepth\@ne
	\beamer@computepref\@itemdepth% sets \beameritemnestingprefix
	\usebeamerfont{itemize/enumerate \beameritemnestingprefix body}%
	\usebeamercolor[fg]{itemize/enumerate \beameritemnestingprefix body}%
	\usebeamertemplate{itemize/enumerate \beameritemnestingprefix body begin}%
	\list
	{\usebeamertemplate{itemize \beameritemnestingprefix item}}
	{\def\makelabel##1{%
			{%
				\hss\llap{{%
						\usebeamerfont*{itemize \beameritemnestingprefix item}%
						\usebeamercolor[fg]{itemize \beameritemnestingprefix item}##1}}%
			}%
		}%
	}
	\fi%
	\beamer@cramped%
	\justifying% NEW
	%\raggedright% ORIGINAL
	\beamer@firstlineitemizeunskip%
}
\makeatother

%\setlength{\parindent}{15mm}
% Hopefully this avoids words being hyphenated
\pretolerance=1000
\tolerance=1000 
\emergencystretch=10pt


\usetheme{InFoMM}
\usepackage{amsmath} % Nice maths symbols.
\usepackage{amssymb} % Nice variable symbols.
\usepackage{array} % Allow for custom column widths in tables.
\usepackage{arydshln} % Dashed lines using \hdashline \cdashline
\usepackage{bbm} % Gives Blackboard fonts.
\usepackage{bm} % Bold math symbols.
\usepackage[margin=10pt,font=small,labelfont=bf,labelsep=endash]{caption} % Caption figures and tables nicely.
\usepackage{enumitem} % Nice listing options in itemize and enumerate.
\usepackage{esdiff} % Gives nice differential operators.
\usepackage{esvect} % Gives nice vector arrows.
\usepackage{float} % Nice figure placement.
\usepackage{geometry} % Use nice margins.
\usepackage{graphicx} % Include figures.
\usepackage[colorlinks=true,linkcolor=blue,urlcolor=blue,citecolor=blue,anchorcolor=blue]{hyperref} % Colour links.
\usepackage{indentfirst} % Indents the first paragraph.
\usepackage{letltxmacro} % For defining a nice SQRT symbol.
%\usepackage{listings} % The listings package for code.
\usepackage[framed,numbered,autolinebreaks,useliterate]{mcode} % Inports Listings package ideal for MATLAB.
\usepackage{multirow} % Nice table cells spanning many rows.
\usepackage{multicol} % If I want to use multiple columns.
\usepackage[numbers, sort&compress]{natbib} % Nice references.
\usepackage{physics} % Nice partial derivatives and BRAKET notation.
\usepackage{subcaption} % Side by side figures.
\usepackage{tikz} % Nice diagrams.
\usepackage{xspace} % Gives nice spacing for commands.


% Present the references in the order they are used.
\bibliographystyle{unsrtnat}

% Listing -> Code in environment labels.
\renewcommand{\lstlistingname}{Code}

% Nice paragraph indents.
\setlength{\parindent}{15mm}

% Giving the refereneces the right title.
\renewcommand{\bibname}{References}

% Gives nice margins.
\geometry{left=20mm, right=20mm, top=20mm, bottom=20mm}

% Removes hyphenation
\tolerance=1
\emergencystretch=\maxdimen
\hyphenpenalty=10000
\hbadness=10000

% To change the spacing in lists
\setlist{noitemsep} % or \setlist{noitemsep} to leave space around whole list
%\setenumerate{itemsep=-0.4em,topsep=0.5em} % Seems to look nice.

% Custom column widths using C{2cm}, L, R, etc.
\newcolumntype{L}[1]{>{\raggedright\let\newline\\\arraybackslash\hspace{0pt}}m{#1}}
\newcolumntype{C}[1]{>{\centering\let\newline\\\arraybackslash\hspace{0pt}}m{#1}}
\newcolumntype{R}[1]{>{\raggedleft\let\newline\\\arraybackslash\hspace{0pt}}m{#1}}

% Gives a nice column separation in multicolumn mode.
\setlength{\columnsep}{5mm}

% Figure environment for use in multicolumn. To put in captions use \captionof{figure}{content of caption}.
\newenvironment{Figure}
{\par\medskip\noindent\minipage{\linewidth}}
{\endminipage\par\medskip}

% Gives the nice SQRT symbol.
\makeatletter
\let\oldr@@t\r@@t
\def\r@@t#1#2{%
	\setbox0=\hbox{$\oldr@@t#1{#2\,}$}\dimen0=\ht0
	\advance\dimen0-0.2\ht0
	\setbox2=\hbox{\vrule height\ht0 depth -\dimen0}%
	{\box0\lower0.4pt\box2}}
\LetLtxMacro{\oldsqrt}{\sqrt}
\renewcommand*{\sqrt}[2][\ ]{\oldsqrt[#1]{#2}}
\makeatother


\title{Simulating networks using Markov evolution}
\date{February 2017}
\author{Oliver Sheridan-Methven}

\newcommand{\clique}{\mathbbm{1}}

\usepackage[percent]{overpic}
\usepackage[export]{adjustbox}


\begin{document}
\maketitle

\begin{frame}{Problem description}
	We simulate the equilibrium states of undirected networks which evolve through a Markov process. The adjacency matrix $ A_n $ of size $ N \times N $  with $ 2 \le N $ evolves in expectation through 
	\begin{equation}
	\bra{A_{n+1}}\ket{A_n} = \left(\clique - \omega(A_n) \right)\circ A_n + (\clique - A_n)\circ \alpha(A_n)
	\end{equation}
	where $ \clique $ represents the clique matrix, $ \omega(A_n) = \bar{\omega} \clique $, and $ \alpha(A_n) = \delta \clique + \epsilon A_n^2 \circ \clique $. As $ n $ becomes very large we have $ \lim_{1 \ll n} \expval{A_n}  \to p_n \clique $ where $ p_n $ evolves through
	\begin{equation}
	\label{eqt:probability_evolution}
	p_{n+1} = (1 - \bar{\omega})p_n + (1-p_n)(\delta + \varepsilon(N-2) p_n^2)
	\end{equation}
	under the constraints $ 0 < \bar{\omega} < 1 $, $ 0 < \delta $, $ 0 < \varepsilon $, and $ 0 < \delta + \varepsilon (N-2) < 1 $.
\end{frame}

\begin{frame}{Steady state equilibria solutions}
	We wish to find the steady state solution to \eqref{eqt:probability_evolution}, which involves finding the roots to 
	\begin{equation}
	\label{eqt:p_roots}
		0 = \delta - (\bar{\omega} + \delta)p + \bar{\varepsilon}p^2 -  \bar{\varepsilon}p^3
	\end{equation}
	where $ \bar{\varepsilon} =\varepsilon(N-2) $. We can construct solutions to this by expressing \eqref{eqt:p_roots} as 
	\begin{equation}
	\label{eqt:p_roots_construction}
	0 = \gamma (\alpha' + \beta' + p^2 - p^3) = \gamma (p - a)(p - b)(p - c)
	\end{equation}
	where we specify some $ 0< a< b< c< 1 $. Equating the coefficients in \eqref{eqt:p_roots} and \eqref{eqt:p_roots_construction} gives 
	\begin{align}
	a + b + c & = 1  \label{eqt:abc_1}\\
	ab + ac + bc & = -\beta' \label{eqt:abc_beta} \\
	abc & = \alpha' \label{eqt:abc_alpha}.
	\end{align}
	We now choose some $ 0< \varkappa < 1$ and set $ \varkappa = \delta + \bar{\varepsilon} $. We substitute \eqref{eqt:abc_1}-\eqref{eqt:abc_alpha} into \eqref{eqt:p_roots} to give
	\begin{align}
	\epsilon & = \dfrac{\varkappa}{(1 + \alpha')(N-2)} \\
	\delta & = \dfrac{\varkappa\alpha'}{1 + \alpha'} \\
	\bar{\omega} & = -\dfrac{\varkappa(\beta' + \alpha')}{1 + \alpha'}
	\end{align}
	where it remains to verify \emph{a posteriori} that $ p_{n+1} \in [0, 1] \quad \forall p_n \in [0, 1]$.
\end{frame}


\begin{frame}{Simulations}
\begin{figure}[t]
	\centering
\begin{overpic}[height=62mm]{comparison_of_prediction_and_simulation} 
\put(54,38){\includegraphics[height=20.5mm, trim={20mm 13mm 20mm 13mm}, clip=true, frame]{tunnelling_into_unstable_minimal}}
\end{overpic}	
\caption{Simulations of $ A \in \mathbb{R}^{200\times 200}$ matrices using $ \varepsilon \approx 0.0100 $, $ \bar{\omega} \approx 0.265 $, and $ \delta \approx 0.0196 $. Stable equilibria are at $ a = 0.1$, $  c = 0.5 $, and an instability at $ b=0.4 $, where $ \Gamma $ passes through $ b $. Solid lines are simulations of $ A $, and dotted lines are the expected trajectories from the initial conditions. (\textbf{Inset}) Simulations using $ A \in \mathbb{R}^{100\times 100}$.}
\label{fig:simulation}
\end{figure}
\end{frame}

\begin{frame}{items}

    	\begin{itemize}
    		\setlength\itemsep{1em}
    		\item Fourier transforms are used to analyse the frequency of incoming signals.
    		\item Different modulation techniques express the signal encoded in the carrier wave with a prescribed baud rate.
    		\item DFT is used to handle the signals programatically.
    		\item Spectrogram is a useful tool to visualise how frequencies change in time for modulated signals.
    		\item We have explored different ways to detect the baud rade of an unknown signal.
    	\end{itemize}
\end{frame}


\end{document}