\documentclass{beamer}

\usetheme{InFoMM}
\usepackage{amsmath} % Nice maths symbols.
\usepackage{amssymb} % Nice variable symbols.
\usepackage{array} % Allow for custom column widths in tables.
\usepackage{arydshln} % Dashed lines using \hdashline \cdashline
\usepackage{bbm} % Gives Blackboard fonts.
\usepackage{bm} % Bold math symbols.
\usepackage[margin=10pt,font=small,labelfont=bf,labelsep=endash]{caption} % Caption figures and tables nicely.
\usepackage{enumitem} % Nice listing options in itemize and enumerate.
\usepackage{esdiff} % Gives nice differential operators.
\usepackage{esvect} % Gives nice vector arrows.
\usepackage{float} % Nice figure placement.
\usepackage{geometry} % Use nice margins.
\usepackage{graphicx} % Include figures.
\usepackage[colorlinks=true,linkcolor=blue,urlcolor=blue,citecolor=blue,anchorcolor=blue]{hyperref} % Colour links.
\usepackage{indentfirst} % Indents the first paragraph.
\usepackage{letltxmacro} % For defining a nice SQRT symbol.
%\usepackage{listings} % The listings package for code.
\usepackage[framed,numbered,autolinebreaks,useliterate]{mcode} % Inports Listings package ideal for MATLAB.
\usepackage{multirow} % Nice table cells spanning many rows.
\usepackage{multicol} % If I want to use multiple columns.
\usepackage[numbers, sort&compress]{natbib} % Nice references.
\usepackage{physics} % Nice partial derivatives and BRAKET notation.
\usepackage{subcaption} % Side by side figures.
\usepackage{tikz} % Nice diagrams.
\usepackage{xspace} % Gives nice spacing for commands.


% Present the references in the order they are used.
\bibliographystyle{unsrtnat}

% Listing -> Code in environment labels.
\renewcommand{\lstlistingname}{Code}

% Nice paragraph indents.
\setlength{\parindent}{15mm}

% Giving the refereneces the right title.
\renewcommand{\bibname}{References}

% Gives nice margins.
\geometry{left=20mm, right=20mm, top=20mm, bottom=20mm}

% Removes hyphenation
\tolerance=1
\emergencystretch=\maxdimen
\hyphenpenalty=10000
\hbadness=10000

% To change the spacing in lists
\setlist{noitemsep} % or \setlist{noitemsep} to leave space around whole list
%\setenumerate{itemsep=-0.4em,topsep=0.5em} % Seems to look nice.

% Custom column widths using C{2cm}, L, R, etc.
\newcolumntype{L}[1]{>{\raggedright\let\newline\\\arraybackslash\hspace{0pt}}m{#1}}
\newcolumntype{C}[1]{>{\centering\let\newline\\\arraybackslash\hspace{0pt}}m{#1}}
\newcolumntype{R}[1]{>{\raggedleft\let\newline\\\arraybackslash\hspace{0pt}}m{#1}}

% Gives a nice column separation in multicolumn mode.
\setlength{\columnsep}{5mm}

% Figure environment for use in multicolumn. To put in captions use \captionof{figure}{content of caption}.
\newenvironment{Figure}
{\par\medskip\noindent\minipage{\linewidth}}
{\endminipage\par\medskip}

% Gives the nice SQRT symbol.
\makeatletter
\let\oldr@@t\r@@t
\def\r@@t#1#2{%
	\setbox0=\hbox{$\oldr@@t#1{#2\,}$}\dimen0=\ht0
	\advance\dimen0-0.2\ht0
	\setbox2=\hbox{\vrule height\ht0 depth -\dimen0}%
	{\box0\lower0.4pt\box2}}
\LetLtxMacro{\oldsqrt}{\sqrt}
\renewcommand*{\sqrt}[2][\ ]{\oldsqrt[#1]{#2}}
\makeatother

\newcommand{\diff}[3][]{\frac{d^{#1}#2}{d{#3}^{#1}}}
\newcommand{\pdiff}[3][]{\frac{\partial^{#1}#2}{\partial{#3}^{#1}}}

\title{A New Method of Modelling Tuneable Lasers with Functional Composition}
\date{October 2019}
\author{Brady Metherall}

\begin{document}
\frame{\titlepage}

\section{Introduction}

\frame{
\frametitle{Laser Overview}
\begin{itemize}
\item The word laser was originally an acronym for Light Amplification by Stimulated Emission of Radiation
\item The defining feature of laser light is coherence, where the peaks and trough overlap causing very strong constructive interference
\item Typical lasers, such as Helium-Neon gas lasers or laser pointers, are monochromatic (operate at a single wavelength---have a very narrow bandwidth)
\end{itemize}
}

\frame{
\frametitle{Tuneable Lasers}
Tuneable lasers
\begin{itemize}
\item have a much wider bandwidth (up to $\sim 100$ nm)
\item lase continuously at all of these wavelengths
\item have applications in spectroscopy and high resolution imaging
%\item use mode-locking---a method to create ultrashort laser pulses ($10^{-12}$--$10^{-15}$ s)
\end{itemize}
}

\section{Modelling Efforts}

\frame{
\frametitle{Nonlinear Optics}
The standard equation for studying nonlinear optics is the generalized nonlinear Schr\"odinger equation,
\begin{align}
\pdiff{A}{z} &= \underbrace{- i \frac{\beta_2}{2}\pdiff[2]{A}{T}}_{\text{Dispersion}} + \underbrace{i \gamma |A|^2 A}_{\text{Nonlinearity}} + \underbrace{\underbrace{\frac{1}{2}g(A) A}_{\text{Gain}} - \underbrace{\alpha A}_{\text{Loss}}}_{\text{Generalization}}.
\label{eq:gnlse}
\end{align}
\begin{itemize}
\item Derived from the nonlinear wave equation
\item Uses comoving coordinates so that the reference frame propagates with the pulse at the group velocity
\begin{align*}
T = t - \frac{z}{v_g}
\end{align*}
\end{itemize}
}

\end{document}